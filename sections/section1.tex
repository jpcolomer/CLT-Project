The paper we studied is ``Knows What It Knows: A Framework For Self-Aware Learning''\cite{KWIK}.
This paper introduces a new learning framework called KWIK (Know What It Knows) that has some common elements with 
the learning frameworks covered in the course: PAC and OLMB.\\

A KWIK algorithm returns a prediction only when it has certainty that is the correct output for the given input. Otherwise, it outputs
``I don't know`` ($\bot$). Additionally, there is a polynomial upper bound on the number of times a KWIK algorithm outputs $\bot$.\\

KWIK was motivated by the idea of distinguishing predictions that have been learned with high accuracy from the others. \\

The structure of this document is the following:
\begin{itemize}
  \item Section 2 provides a definition of the KWIK model.
  \item Section 3 motivates the study of KWIK models by using an example.
  \item Section 4 shows a set of concept classes that can be KWIK-learned.
  \item Section 5 shows different ways of combining KWIK algorithms in order to learn a more complex concept class
\end{itemize}

% KWIK es muy guay y está relacionado con los que vimos en clase
The paper we studied introduces the KWIK model, which, instead of allowing
incorrect outputs, it returns ``I don't know'' for inputs for which it is not
sure about the correct output. It also presented a set of concept classes that
are KWIK-learnable, introducing algorithms for learning them and giving
specific examples of their executions. \\

% Paper bien escrito, pero encontramos fallos que nos hicieron struggle
Overall, the paper is clear and precise in its explanations. However, we found
some errors that misled us and we had to spend some time trying to figure out
the actual values that were incorrect in the paper. In addition to this, we
found it very hard to understand the proof in Algorithm 9 which shows that $l_i$
provides a robust measure for eliminating an invalid predictor. \\

We found it useful to relate the KWIK model with the models we studied in class.
It is related to the OLMB and PAC learning models we learned in class. It has
adversarial inputs like the OLMB model, and its output is always
$\varepsilon$-accurate with probability at least $1 - \delta$, like the PAC
model. \\
